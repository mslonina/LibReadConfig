\hypertarget{sample-readconfig_8c}{
\subsection{/home/slonina/LibReadConfig/doc/sample-\/readconfig.c File Reference}
\label{sample-readconfig_8c}\index{/home/slonina/LibReadConfig/doc/sample-\/readconfig.c@{/home/slonina/LibReadConfig/doc/sample-\/readconfig.c}}
}


This is a sample file for using LibReadConfig.  
{\ttfamily \#include $<$stdio.h$>$}\par
{\ttfamily \#include $<$stdlib.h$>$}\par
{\ttfamily \#include $<$unistd.h$>$}\par
{\ttfamily \#include $<$stdarg.h$>$}\par
{\ttfamily \#include $<$sys/stat.h$>$}\par
{\ttfamily \#include $<$sys/types.h$>$}\par
{\ttfamily \#include $<$sys/wait.h$>$}\par
{\ttfamily \#include $<$signal.h$>$}\par
{\ttfamily \#include $<$fcntl.h$>$}\par
{\ttfamily \#include $<$sys/dir.h$>$}\par
{\ttfamily \#include $<$dirent.h$>$}\par
{\ttfamily \#include $<$ctype.h$>$}\par
{\ttfamily \#include $<$string.h$>$}\par
{\ttfamily \#include \char`\"{}libreadconfig.h\char`\"{}}\par
{\ttfamily \#include \char`\"{}hdf5.h\char`\"{}}\par


\subsubsection{Detailed Description}
This is a sample file for using LibReadConfig. Using LibReadConfig is as simple as possible. This example will show you how to do that. 

Definition in file \hyperlink{sample-readconfig_8c_source}{sample-\/readconfig.c}.