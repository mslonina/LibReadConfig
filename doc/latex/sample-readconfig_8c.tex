\hypertarget{sample-readconfig_8c}{
\subsection{/home/mariusz/DOKTORAT/CODE/LibReadConfig/doc/sample-readconfig.c File Reference}
\label{sample-readconfig_8c}\index{/home/mariusz/DOKTORAT/CODE/LibReadConfig/doc/sample-readconfig.c@{/home/mariusz/DOKTORAT/CODE/LibReadConfig/doc/sample-readconfig.c}}
}
This is a sample file for using LibReadConfig.  


{\tt \#include $<$stdio.h$>$}\par
{\tt \#include $<$stdlib.h$>$}\par
{\tt \#include $<$unistd.h$>$}\par
{\tt \#include $<$stdarg.h$>$}\par
{\tt \#include $<$sys/stat.h$>$}\par
{\tt \#include $<$sys/types.h$>$}\par
{\tt \#include $<$sys/wait.h$>$}\par
{\tt \#include $<$signal.h$>$}\par
{\tt \#include $<$fcntl.h$>$}\par
{\tt \#include $<$sys/dir.h$>$}\par
{\tt \#include $<$dirent.h$>$}\par
{\tt \#include $<$ctype.h$>$}\par
{\tt \#include $<$string.h$>$}\par
{\tt \#include \char`\"{}libreadconfig.h\char`\"{}}\par
{\tt \#include \char`\"{}hdf5.h\char`\"{}}\par


\subsubsection{Detailed Description}
This is a sample file for using LibReadConfig. 

Using LibReadConfig is as simple as possible. This example will show you how to do that. 

Definition in file \hyperlink{sample-readconfig_8c-source}{sample-readconfig.c}.